\PassOptionsToPackage{unicode=true}{hyperref} % options for packages loaded elsewhere
\PassOptionsToPackage{hyphens}{url}
%
\documentclass[]{article}
\usepackage{lmodern}
\usepackage{amssymb,amsmath}
\usepackage{ifxetex,ifluatex}
\usepackage{fixltx2e} % provides \textsubscript
\ifnum 0\ifxetex 1\fi\ifluatex 1\fi=0 % if pdftex
  \usepackage[T1]{fontenc}
  \usepackage[utf8]{inputenc}
  \usepackage{textcomp} % provides euro and other symbols
\else % if luatex or xelatex
  \usepackage{unicode-math}
  \defaultfontfeatures{Ligatures=TeX,Scale=MatchLowercase}
\fi
% use upquote if available, for straight quotes in verbatim environments
\IfFileExists{upquote.sty}{\usepackage{upquote}}{}
% use microtype if available
\IfFileExists{microtype.sty}{%
\usepackage[]{microtype}
\UseMicrotypeSet[protrusion]{basicmath} % disable protrusion for tt fonts
}{}
\IfFileExists{parskip.sty}{%
\usepackage{parskip}
}{% else
\setlength{\parindent}{0pt}
\setlength{\parskip}{6pt plus 2pt minus 1pt}
}
\usepackage{hyperref}
\hypersetup{
            pdftitle={Final Report},
            pdfauthor={Carleena Ortega and Saelin Bjornson},
            pdfborder={0 0 0},
            breaklinks=true}
\urlstyle{same}  % don't use monospace font for urls
\usepackage[margin=1in]{geometry}
\usepackage{graphicx,grffile}
\makeatletter
\def\maxwidth{\ifdim\Gin@nat@width>\linewidth\linewidth\else\Gin@nat@width\fi}
\def\maxheight{\ifdim\Gin@nat@height>\textheight\textheight\else\Gin@nat@height\fi}
\makeatother
% Scale images if necessary, so that they will not overflow the page
% margins by default, and it is still possible to overwrite the defaults
% using explicit options in \includegraphics[width, height, ...]{}
\setkeys{Gin}{width=\maxwidth,height=\maxheight,keepaspectratio}
\setlength{\emergencystretch}{3em}  % prevent overfull lines
\providecommand{\tightlist}{%
  \setlength{\itemsep}{0pt}\setlength{\parskip}{0pt}}
\setcounter{secnumdepth}{0}
% Redefines (sub)paragraphs to behave more like sections
\ifx\paragraph\undefined\else
\let\oldparagraph\paragraph
\renewcommand{\paragraph}[1]{\oldparagraph{#1}\mbox{}}
\fi
\ifx\subparagraph\undefined\else
\let\oldsubparagraph\subparagraph
\renewcommand{\subparagraph}[1]{\oldsubparagraph{#1}\mbox{}}
\fi

% set default figure placement to htbp
\makeatletter
\def\fps@figure{htbp}
\makeatother


\title{Final Report}
\author{Carleena Ortega and Saelin Bjornson}
\date{15/03/2020}

\begin{document}
\maketitle

{
\setcounter{tocdepth}{4}
\tableofcontents
}
\hypertarget{adult-income}{%
\subsection{Adult Income}\label{adult-income}}

\hypertarget{introduction}{%
\subsection{Introduction}\label{introduction}}

\hypertarget{the-dataset}{%
\paragraph{The Dataset}\label{the-dataset}}

Who: The data set was extracted by Barry Becker from the 1994 Census
database and is donated by Silicon Graphics\\
What: This is a multivariate dataset with categorical and integer
variables. It contains the predicted income of individuals from the
census with attributes including age, marital status, work class,
education, sex, and race.\\
When: The data is from a 1994 census.\\
Why: The data set is found in the University of California Irvine
Machine Learning Repository, and was used for ML prediction of whether a
person makes over or under 50K a year based on their attributes.\\
How: The census data was collected by survey.

\hypertarget{the-research-questions}{%
\paragraph{The Research Questions}\label{the-research-questions}}

Are the number of hours someone works per week correlated with their
age, relationship, education level or sex?

\hypertarget{exploratory-data-analysis}{%
\subsection{Exploratory Data Analysis}\label{exploratory-data-analysis}}

In this section, we will get to know our dataset better by exploring the
relationship between certain factors.

\hypertarget{age-and-sex}{%
\paragraph{Age and Sex}\label{age-and-sex}}

The plot below shows that there are more male employees than female
employees and that the majority of working males are older than working
females since the male (blue curve) have a peak shifted to the right
with respect to female (red peak).

\includegraphics{../images/Plot_1_Distribution_of_Age_by_Sex.png}

\hypertarget{educational-leve-and-income}{%
\paragraph{Educational Leve and
Income}\label{educational-leve-and-income}}

We observe from the following graphs that a majority of individuals
earning greater than \$50,000 a year only accomplished high school
irrespective of sex or ethnical background.

\includegraphics{../images/Plot_2_Education_Level_of_50K_or_more_Earners.png}

\hypertarget{number-of-work-hours-and-age}{%
\paragraph{Number of Work Hours and
Age}\label{number-of-work-hours-and-age}}

We can deduce from the graph below that individuals work the most hours
between their 40's and 60's (probably full time at 40 hours or more a
week) and that employees under 20 and over 80 years of age work the same
number of hours (probably part time at 25 hours)

\includegraphics{../images/Plot_3_Hours_worked_per_week_by_age.png}

\hypertarget{marital-status-and-number-of-hours-worked}{%
\paragraph{Marital Status and Number of Hours
Worked}\label{marital-status-and-number-of-hours-worked}}

The plot below shows that the working hours between married individuals
and single employees are similar.

\includegraphics{../images/Plot_4_Marital_Status_and_Work_Hours.png}

\hypertarget{analysis}{%
\subsection{Analysis}\label{analysis}}

We have performed linear regression of hours per week vs.~each variable
separately, as well as linear regression using all these variables
together.

This was done using the lm function of the purr package. For example:
lm(hours\_per\_week\textasciitilde{}education,data)

For categorical variables sex, education and relationship, the intercept
is the defaul reference group, where the ``estimate'' is the mean of
that group, and the ``estimates'' of all other variables are the
differences in means between that group and the reference. The statistic
is the t-statistic comparing these means, with a given p-value reporting
the significance of this difference.

\hypertarget{results}{%
\subsection{Results}\label{results}}

\hypertarget{hours-vs.relationship}{%
\paragraph{Hours vs.~Relationship}\label{hours-vs.relationship}}

\begin{verbatim}
## # A tibble: 6 x 5
##   term                       estimate std.error statistic   p.value
##   <chr>                         <dbl>     <dbl>     <dbl>     <dbl>
## 1 (Intercept)                   44.1      0.102     432.  0.       
## 2 relationshipNot-in-family     -3.52     0.164     -21.4 3.32e-101
## 3 relationshipOther-relative    -7.11     0.389     -18.3 1.60e- 74
## 4 relationshipOwn-child        -10.9      0.194     -55.9 0.       
## 5 relationshipUnmarried         -5.02     0.225     -22.3 1.04e-109
## 6 relationshipWife              -7.26     0.314     -23.1 1.42e-117
\end{verbatim}

In this case, we are comparing the mean hours worked per week of
husbands (intercept) to each other relationship category. It appears
that husbands work more than any other age group, (as seen by the
negative estimates), with significant p-values in each case.

\hypertarget{hours-vs.sex}{%
\paragraph{Hours vs.~Sex}\label{hours-vs.sex}}

\begin{verbatim}
## # A tibble: 2 x 5
##   term        estimate std.error statistic p.value
##   <chr>          <dbl>     <dbl>     <dbl>   <dbl>
## 1 (Intercept)    36.4      0.116     314.        0
## 2 sexMale         6.02     0.142      42.5       0
\end{verbatim}

From this output, we can see that the average hours worked per week for
women is 36.1, and men work 6 more hours per week on average.

\hypertarget{hours-vs.age}{%
\paragraph{Hours vs.~Age}\label{hours-vs.age}}

\begin{verbatim}
## # A tibble: 2 x 5
##   term        estimate std.error statistic  p.value
##   <chr>          <dbl>     <dbl>     <dbl>    <dbl>
## 1 (Intercept)  38.0      0.205       186.  0.      
## 2 age           0.0622   0.00500      12.4 2.01e-35
\end{verbatim}

Age

\hypertarget{hours-vs.education}{%
\paragraph{Hours vs.~Education}\label{hours-vs.education}}

\begin{verbatim}
## # A tibble: 16 x 5
##    term                  estimate std.error statistic  p.value
##    <chr>                    <dbl>     <dbl>     <dbl>    <dbl>
##  1 (Intercept)             37.1       0.397    93.4   0.      
##  2 education11th           -3.13      0.531    -5.89  4.01e- 9
##  3 education12th           -1.27      0.704    -1.81  7.10e- 2
##  4 education1st-4th         1.20      1.02      1.19  2.36e- 1
##  5 education5th-6th         1.85      0.773     2.39  1.70e- 2
##  6 education7th-8th         2.31      0.620     3.73  1.90e- 4
##  7 education9th             0.992     0.665     1.49  1.36e- 1
##  8 educationAssoc-acdm      3.45      0.543     6.36  2.09e-10
##  9 educationAssoc-voc       4.56      0.513     8.88  7.05e-19
## 10 educationBachelors       5.56      0.430    12.9   3.33e-38
## 11 educationDoctorate       9.92      0.716    13.9   1.57e-43
## 12 educationHS-grad         3.52      0.414     8.51  1.78e-17
## 13 educationMasters         6.78      0.492    13.8   4.70e-43
## 14 educationPreschool      -0.405     1.74     -0.233 8.16e- 1
## 15 educationProf-school    10.4       0.642    16.2   1.69e-58
## 16 educationSome-college    1.80      0.421     4.27  1.94e- 5
\end{verbatim}

In this analysis, the default reference group (intercept) is a 10th
grade education. It appears that those with an 11th grade education work
3 hours less (significant p-value), whereas all other with no more than
a high school education work the same amount (no significant p-values).

Every other education level higher than highschool worked significantly
more hours, as seen by positive estimates of each group and low
p-values.

\hypertarget{all-variables}{%
\paragraph{All Variables}\label{all-variables}}

\begin{verbatim}
## # A tibble: 6 x 5
##   term             estimate std.error statistic   p.value
##   <chr>               <dbl>     <dbl>     <dbl>     <dbl>
## 1 (Intercept)        40.6       0.496    81.9   0.       
## 2 sexMale             4.04      0.179    22.6   1.23e-112
## 3 education11th      -2.26      0.504    -4.49  7.31e-  6
## 4 education12th      -0.536     0.668    -0.801 4.23e-  1
## 5 education1st-4th    0.305     0.964     0.317 7.51e-  1
## 6 education5th-6th    0.575     0.734     0.783 4.34e-  1
\end{verbatim}

\hypertarget{discussion}{%
\subsection{Discussion}\label{discussion}}

Overall we believe a there is a more informative way to analyze our
research question than the default lm() parameters.

Our data is quite biased in that there are much more men (21790) than
women (10771) in data set. This is especially apparent at older ages,
for instance there are 742 women over 60 and 1590 men over 60.

\hypertarget{conclusion}{%
\subsection{Conclusion}\label{conclusion}}

More appropriate linear regression analysis need to be done

\end{document}
